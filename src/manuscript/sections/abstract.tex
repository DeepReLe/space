The families of solutions for a particular motor task  (i.e., feasible activation sets) are high-dimensional subspaces with a well defined structure that emerges naturally from the interactions among the feasible neural commands, the anatomy of the limb, and the mechanical constraints defining the task.
Characterizing their structure, however, has proven challenging.
Here we present a novel computationally efficient approach to characterize their multi-dimensional structure by uniformly sampling their interior using the Hit-and-Run algorithm---a generalization of a discrete Markov chain.
We studied 3D static force production by a realistic model of the human index finger with 7 muscles and 4 kinematic degrees of freedom.
For each of 9 sub-maximal magnitudes of static fingertip force in a given direction, the feasible activation set is a 4-dimensional convex polytope  embedded in 7-dimensional activation space.
We describe the structure of each feasible activation set by the histograms of feasible activations for individual muscles.  Then, we describe the multi-dimensional interaction among these valid muscle activations and six cost functions using an interactive parallel coordinates system. 
This first description of the multi-dimensional nature of families of feasible solutions  has important consequences to our understanding of the neural control of redundant musculature.  For example, the bounding box of the feasible activation set singularly misconstrues the families of feasible activations—and the modes of the histograms for low magnitudes do not necessarily correspond to those for higher ones. Similarly, exploring and exploiting families of feasible solutions is likely more biologically plausible than searching for unique optimal solutions, and knowing these families will help mitigate biomechanical confounds in dimensionality reduction techniques seeking to extract synergies of neural origin.
More importantly, describing the structure of feasible activation sets as raw or cost-weighted multi-dimensional probability distributions marries neuromechanical and Bayesian perspectives into an integrative probabilistic approach to motor control, dysfunction, rehabilitation, and learning/adaptation for neuromechanically realistic limbs.
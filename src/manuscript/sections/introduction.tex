Muscle redundancy is the term used to describe the classical concept that multiple muscle coordination patterns can   produce a same set of joint torques for limb function.
Numerical optimization is a reasonable and useful metaphor of how the nervous system selects and implements muscle coordination patterns. It allows one to select one from among   many  options based on the  different energetic, physiological, and neuromechanical consequences associated with each muscle activation pattern\cite{Chao1978Graphical,Prilutsky2000Muscle,scott2004optimal,todorov2002optimal,crowninshield1981physiologically,higginson2005simulated}. 
Some of the criticisms of numerical optimization, however,  include that it is not necessarily  realistic (i.e., does the nervous system really solve quadratic equations and enforce optimality?), and may even be paradoxical with neural computation, evolutionary biology and clinical reality (i.e., why does disability arise with the damage of even a few muscles?) \cite{valero2015fundamentals,deRugy2012habitual,loeb2000overcomplete,loeb2012optimal}. Nevertheless,  numerical optimization is widely used because of its conceptual appeal,  computational tractability,  efficiency, and practical implementation in computer packages. This has produced a large body of literature containing multiple examples of the use of biologically reasonable---yet at times hotly debated---cost functions  to find unique optimal solutions, even for  limbs with numerous muscles (i.e., high-dimensional problems) \cite{valero2009computational}.

An alternative  approach  focuses on characterizing groups of multiple valid solutions, instead of individual optimal solutions. This approach stems from longtime observations that the variables used to describe functional features (e.g., joint angles) in general, and muscle activations (e.g., EMG signals) in particular, occupy  a lower-dimensional  space than the native dimensionality of all variables (i.e., all joints, all muscles, etc.)  \cite{scholz1999uncontrolled,bizzi2013neural,davella2005shared,Clewley2008Estimating}.   Subsequent work phrased these observations more broadly as a consequence of there being  \emph{families} of feasible muscle activations  (i.e., feasible activation sets) whose well-defined structure that emerges naturally from the interactions among the anatomy of the limb, neural constraints in the activation of muscles, and the mechanical  constraints defining the task\cite{kutch2011muscle,sohn2013cat_bounding_box,Valero-Cuevas1998Large,Valero-Cuevas2015high-dimensional,Kuo1993Human,valero-cuevas2015fundamentals,schieber2004hand}. From this perspective, the fact that EMG signals  exhibit low-dimensionality is to be expected as  part and parcel of meeting the neuromechanical constrains of a task. Thus, research on the nature of muscle coordination should  focus on how nervous system  explores, inhabits, and exploits those sets of feasible solutions \cite{kutch2012challenges,steele2013number,bizzi2013neural,tresch2009case,dingwell2010walkingvariability,racz2013spatiotemporal,steele2015consequences}.

While potentially more biologically realistic than optimization methods,  finding families of feasible solutions  requires  linear and nonlinear geometric methods that are particularly susceptible to computational difficulties in higher dimensions, which has impeded their development and  widespread adoption \cite{valero2009computational,Chao1978Graphical,spoor1983balancing,Kuo1993Human,theodorou2010optimalityEMBC,scholz1999uncontrolled,dingwell2010walkingvariability}. In response to this need,  we present the development of a conceptual framework that enables the computational tractable and efficient description of the structure of feasible  activation sets  for high-dimensional systems. We start from the fundamental notion that, for some problems in motor control involving the production of static forces, the feasible activation set is a convex polytope embedded in the native high-dimensional space of all muscles (the general case is  considered in the Discussion).  By efficiently sampling from the large-dimensional feasible activation sets, we are  able to describe their structure using the  uni- and multi-dimensional statistical properties of the muscle coordination patterns that constitute them.  We then go on to discuss how  we can use the probabilistic structure of  feasible activations set to bridge  from neuromechanics to  Bayesian and stochastic  exploration-exploitation methods as a biologically plausible model of sensorimotor function.



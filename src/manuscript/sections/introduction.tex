
\section{INTRODUCTION}

Muscle redundancy is the term used to describe the underdetermined nature of neural control of musculature.
The classical notion of muscle redundancy  proposes that, faced with an infinite number of possible muscle activation patterns for a given task, the nervous system optimizes in some fashion to select one solution.
Here, each of $N$ muscles represents a dimension of control on an end effector, and at any moment of a task, a muscle activation pattern exists as a point in $[0,1]^N$,--- the $n$-dimensional hypercube --- where each muscle's maximal activation is normalized to $1$ \cite{Valero-Cuevas1998Large}.
Thus researchers often seek to infer the optimization approach and the cost functions the nervous system utilizes to select effective points in activation space to produce natural behavior \cite{Chao1978Graphical,Prilutsky2000Muscle,scott2004optimal,todorov2002optimal,crowninshield1981physiologically,higginson2005simulated}. 

Implicit in these optimization procedures is the notion that there exists a well structured set of feasible solutions. Thus several of us have focused on describing and understanding those high-dimensional subspaces  embedded in $[0,1]^N$ \cite{kutch2011muscle,kutch2012challenges,sohn2013cat_bounding_box,Valero-Cuevas1998Large,Valero-Cuevas2015high-dimensional}.

For the case of static force production with a limb, the muscle redundancy problem is phrased in computational geometry: find the structure of the set of all feasible muscle activations, given the limb mechanics and the task constraints \cite{avis1992Pivoting,Valero-Cuevas1998Large,Valero-Cuevas2009mathematical,Valero-Cuevas2015high-dimensional}. We aim to explore what the solution space looks like, and uncover the structure of the feasible activation space for given static force tasks.
If each muscle's maximal activation is normalized to 1, the constraint $\textbf{a} \in [0,1]^n$ describes that the feasible activation space lies in the $n$-dimensional unit hypercube (also called the $n$-cube).%; the activation space for the index finger is within the unit $7$-cube.

\subsection{High dimensionality difficulties}
Consider a model of a static fingertip force, with 7 muscles arculating the index finger's 4 degrees of freedom (DoF), which will be further discribed in Section \ref{ss:finger}.
Assuming independent control of each muscle (non-synergistic model), each muscle has a unique force vector at the endpoint; the end effector has 7 unique vectors it can linearly combine to generate any vector of static force.
This yields a unit $7$-cube in charge of producing a 4-dimensional output wrench.
On order to uncover the structure and relationship of these spaces, we cannot visualize all dimensions simultaneously. %as we could with a simple 3-muscle model.

The solution of the above system is a convex polytope is called the \emph{feasible activation set} (see Section \ref{s:methods}). To date, the structure of this high-dimensional polytope is inferred by its bounding box  \cite{kutch2011muscle,sohn2013cat_bounding_box,Valero-Cuevas2015high-dimensional}.  But the bounding box of a convex polytope excludes the details of its shape, thereby precluding comparison, since the polytope is a lower dimensional object embedded into $[0,1]^n$.  Empirical dimensionality-reduction methods have also been used to calculate a basis vectors for such subspaces \cite{Clewley2008Estimating,davella2005shared,krishnamoorthy2003muscle}. But those basis  vectors only provide a description of the dimension, orientation, and aspect ratio of the polytope, but not of its boundaries or internal  structure.

Here we present a novel application of the well-known Hit-and-Run algorithm \cite{smith1984efficient} to describe the internal structure of these high-dimensional feasible activation sets. The input to our hit-and-run procedure requires a task force, along with the system's endpoint Jacobian, maximal tendon forces, and a moment arm matrix.

We applied our approach to two separate musculoskeletal models:

1. A fabricated schematic system, which we designed to have three muscles articulating one DOF, and one dimension of output force.

2. A realistic model, with seven muscles articulating four DOFs, and four dimensional output force \cite{Valero-Cuevas1998Large}.

With this, below are the key ideas and findings we present with this paper:

\begin{itemize}
\item {The high-dimensional space serves as a =Bayesian posterior probability distribution, with a nonparametric, piecewise shape.}
\item {For some muscles, we found that the bounding box exceptionally misconstrues the actual shape of the feasible activation space.}
\item {Hit-and-run sampling of the solution space is computationally tractable; fewer than ten thousand points uncovered the shape of the distributions.}
\item {Our approach provides a more granular context to the space within which the central nervous system optimizes.}
\item {We apply six different cost functions (post-hoc) to all solutions, thereby providing spatial context to where 'optimal' solutions lie within the space.}
\item {We designed an interactive parallel coordinates platform for visualizing and manipulating constraints to the solution space, such as muscle dysfunction, muscle hyperactivity, as well as constraining the upper and lower bounds for six different cost functions. We can compare cost functions side-by-side and view subsets of the dataset after applying cost function constraints. }
\end{itemize}


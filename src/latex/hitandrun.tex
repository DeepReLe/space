% -----------------------
% ///--- Preamble --- \\\
% -----------------------

\documentclass[a4paper,11pt]{article}

% --- Packages ---
% Language
\usepackage[utf8]{inputenc}
\usepackage[T1]{fontenc}
\usepackage[english]{babel}
% Font
\usepackage{lmodern}
% Math
\usepackage{amsmath,amssymb,amsfonts,mathrsfs}
\usepackage[amsmath,thmmarks]{ntheorem}
% Code
%\usepackage{listings}
%\usepackage{color}
%\lstset{language=Octave,
%        showspaces=false,
%        showstringspaces=false,
%        frame=single,
%        basicstyle=\scriptsize,
%        numbers=left,
%        numberstyle=\footnotesize,
%        numbersep=5pt}
% Graphics (Postscript)
%\usepackage{graphicx, pst-all}
\usepackage{graphicx}
% Style
\usepackage{fancyhdr}
\usepackage{lastpage}

%Commands
\renewcommand{\labelenumi}{(\alph{enumi})}
\renewcommand{\labelenumii}{(\arabic{enumii})}

% --- Definitions ---
% Counter
\newcounter{behctr}
% Theorem-Styles
\newtheorem{claim}[behctr]{Claim}
\newtheorem{rmk}{Remark}
%% Proof environment with a small square as a "qed" symbol
\theoremstyle{nonumberplain}
\newtheorem{defn}{Definition}
\newtheorem{thm}{Theorem}
%\theoremheaderfont{\normalfont\slshape}
\theorembodyfont{\normalfont}
\theoremsymbol{\ensuremath{\square}}
\newtheorem{proof}{Proof}
\setlength\theorempostskipamount{0pt}
\setlength\theorempreskipamount{5pt}
% Math-commands
\newcommand{\inv}{^{-1}}  % inverse
\newcommand{\vnorm}[1]{\left|\left|#1\right|\right|}  % norm of a vector
\newcommand{\set}[2]{\{\nonscript\,{#1}\mid{#2}\nonscript\,\}}  % sets with proper spacing
\DeclareMathOperator{\rank}{rk}
\DeclareMathOperator{\id}{id}
\DeclareMathOperator{\Inv}{Inv}
\DeclareMathOperator{\Aut}{Aut}
\DeclareMathOperator{\Ker}{Ker}
\DeclareMathOperator{\Image}{Im}
\DeclareMathOperator{\supp}{supp}
\DeclareMathOperator{\grad}{grad}
\DeclareMathOperator\nN{\mathbb{N}}
\DeclareMathOperator\nZ{\mathbb{Z}}
\DeclareMathOperator\nQ{\mathbb{Q}}
\DeclareMathOperator\nR{\mathbb{R}}
\DeclareMathOperator\nC{\mathbb{C}}
\DeclareMathOperator\nH{\mathbb{H}}
\DeclareMathOperator\nD{\mathbb{D}}
\renewcommand{\vec}[1]{\mathbf{#1}}  % vectors
% Hat for Vectors
%\let\oldhat\hat
%\renewcommand{\hat}[1]{\oldhat{\mathbf{#1}}}
% Postscripsettings
%\psset{arrows=->,
%       unit=1.65cm,
%       arcangle=15,
%       linewidth=0.65pt,
%       arrowsize=4pt,
%       arrowinset=.25,
%       labelsep=2.5pt,
%       nodesep=2pt,
%       subgriddiv=10,
%       shortput=nab}

% --- Layout ---
\pagestyle{fancy}
% Text
\addtolength{\textwidth}{2cm}
\addtolength{\oddsidemargin}{-1cm}
\addtolength{\evensidemargin}{-1cm}
\addtolength{\textheight}{1.5cm}
\addtolength{\headheight}{2pt}
\addtolength{\topmargin}{-0.5cm}
\addtolength{\headsep}{-0.25cm}
% Header
\renewcommand{\headrulewidth}{0.4pt}
\fancyheadoffset{0.5cm}
%\lhead{D-MATH}
%\chead{Computational Geometry HS12}
%\rhead{May Szedl\'ak}
% Footer
\renewcommand{\footrulewidth}{0.4pt}
\fancyfootoffset{-4cm}
\lfoot{}
\cfoot{\thepage\ / \pageref{LastPage}}
\rfoot{}
\parindent0pt

% -----------------------
% ///--- Document --- \\\
% -----------------------
\begin{document}

\section{Abstract}
300 Words

Abstract
------PLOS Computational Biology-----
The abstract of the paper should be succinct; it must not exceed 300 words. Authors should mention the techniques used without going into methodological detail and should summarize the most important results. While the abstract is conceptually divided into three sections (Background, Methodology/Principal Findings, and Conclusions/Significance), please do not apply these distinct headings to the abstract within the article file. Please do not include any citations and avoid specialist abbreviations.
-----------


What we want to achieve with this abstract:
BRIAN - Introduce linear approximations of muscles for FFS and FAS
MAY- Show how computationally complex it would be to get volume slices of a zonotope in 7D
MAY- Show the benefits of having points
MAY Discuss prediction with points and how its related to volume
MAY -What is Hit-And-Run, and how does it help us solve our problem
Talk about how distributions are not uniform across the dimensions, and the majority of solutions can lie in one cluster.
Briefly touch on the limitations of this model and other models
What does this mean in terms of neural control models made by other researchers in the past?

Things that we achieved:
?
?
?








\section{Hit-and-Run}
The Hit-and-Run algorithm used for sampling in a convex body $K$, was introduced by Smith in 1984 \cite{Smith}. The mixing time is known to be $\mathcal{O}^*(n^2R^2/r^2)$, where $R$ and $r$ are the radii of the inscribed and cicumscribed ball of $K$ respectively \cite{Dyer, Lovasz}.
In the case of the muscles of a limb, we are interested in the polygon $P$ that is given by the set of all possible activations $\textbf{a} \in \mathbb{R}^n$ that satisfy
\[\textbf{f} = A\textbf{a}, \textbf{a} \in [0,1]^n,\]
where $\textbf{f} \in \mathbb{R}^m$ is a fixed force vector and $A = J^{-T}RF_m \in \mathbb{R}^{m \times n}$. $P$ is bounded by the unit $n$-cube since all variables $a_i$, $i \in [n]$ are bounded by 0 and 1 from below, above respectively.

Consider the following $1 \times 3$ example.
\begin{align*}
&1 = \frac{10}{3}a_1 - \frac{53}{15}a_2 + 2a_3 \\
&a_1, a_2, a_3 \in [0,1],
\end{align*}
the set of feasible activations is given by the shaded set in figure ??.

\textit{\textbf{FIGURE 1}}

The Hit-and-Run walk on $P$ is defined as follows (it works analogously for any convex body). 
\begin{enumerate}
\item Find a given starting point $\textbf{p}$ of $P$ .
\item Generate a random direction through $\textbf{p}$ (uniformly at random over all directions).
\item Choose the next point of the sampling algorithm uniformly at random from the segment of the line in $P$. 
\item Repeat from $(b)$ the above steps with the new point as the starting point.
\end{enumerate}

\textit{
\textbf{FIGURE 2}: Figures in 2D or 3D? Better in 3D to have the overview.
\begin{enumerate}
\item figure inner point
\item figure choice of direction
\item figure line segment
\item figure choice of new point
\item figure of distribution after some sampling
\end{enumerate}}

The implementation of this algorithm is straight forward except for the choice of the random direction. How do we sample uniformly at random (u.a.r.\) from all directions in $P$? Suppose that $\textbf{q}$ is a direction in $P$ and $p \in P$. Then by definition of $P$, $\textbf{q}$ must satisfy $\textbf{f} = A(\textbf{p}+\textbf{q})$. Since $\textbf{p} \in P$, we know that $\textbf{f} = A\textbf{p}$ and therefore 
\[\textbf{f} = A(\textbf{p} + \textbf{q}) = \textbf{f} + A\textbf{q}\]
and hence
\[A\textbf{q} = 0.\]

\textit{\textbf{FIGURE 3}: $q$ must be parallel to the plane given by $1 = \frac{10}{3}a_1 - \frac{53}{15}a_2 + 2a_3$.}

We therefore need to choose directions uniformly at random from all directions in the vectorspace 
\[V = \{\textbf{q} \in \mathbb{R}^n | A\textbf{q} = 0\}.\]

As shown by Marsaglia this can be done as follows \cite{Marsaglia}.

\begin{enumerate}
\item
Find an orthonormal basis $b_1, \dots, b_r \in \mathbb{R}^{n}$ of $A\textbf{q} =0$.
\item
Choose $(\lambda_1, \dots, \lambda_r) \in \mathcal{N}(0,1)^n$ (from the Gaussian distribution).
\item
$\sum_{i=1}^r \lambda_i b_i$ is a u.a.r.\ direction.
\end{enumerate}

A basis of a vectorspace $V$ is a minimal set of vectors that generate $V$, and it is orthonormal if the vectors are pairwise orthogonal (perpendicular) and have unit length. Using basic linear algebra one can find a basis for $V = \{A\textbf{q} = 0\}$ and orthogonalize it with the well known Gram-Schmidt method (for details see e.g.\ \cite{Robertson}). Note that in order to get the desired u.a.r.\ distribution the basis needs to be orthonormal. For the limb case we can safely assume that the rows of $A$ are linearly independent and hence the number of basis vectors is $n-m$.

\textit{\textbf{FIGURE 4}
\begin{enumerate}
\item Some basis
\item orthonormal basis
\end{enumerate}}



\bibliographystyle{plain}
\bibliography{redundancyFVC}

\end{document}